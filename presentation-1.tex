% Created by Jiri Kubik, 2022
% '-> based on previous template by: Petr Cizek, 2018

\documentclass[aspectratio=169]{beamer}
\usepackage[utf8]{inputenc}
%\usepackage[T1]{fontenc}  

\usepackage{graphicx}
\usepackage{multimedia}
\usepackage{hyperref}
\usepackage{amsmath}
\usepackage{textcomp}
\usepackage{multirow}
\usepackage{ragged2e}
\usepackage{tabularx}

\usepackage{transparent}

\usepackage{bibentry}
\nobibliography*

% Use template should be called before defining anything else
\usetheme{GyBotStyle}

\author{Jiří Kubík}
\title{Úvod do 3D tisku: motivace, technologie a materiály}
\date{\today}
\institute{%
\vspace{0.25em}
   {\bf \ps}\\ 
\vspace{0.25em}
   {\bf 2022}\\ 
\vspace{0.25em}
   {\GyBot}\\
\vspace{0.25em}
   {rev. 2022-10-1}
}

% uncomment to placing frame title in the header
%\def\compact{1}

% set the left part of the footline to \ComRob
\renewcommand{\footlineleft}{\insertsubsection}
% set the right line of the footline to comrob url
\renewcommand{\footlineright}{\ps}

\input{gybot-commands.tex}

\begin{document}

    \begin{frame}
        \titlepage
    \end{frame}

    \begin{frame}[t]
        \frametitle{Obsah}
        \begin{itemize}
            \item Motivační úvod, aneb co můžeme dělat s 3D tiskem
            \item Organizace semináře a klasifikace
            \item Přehled technologií 3D tisku
            \item Technologie FFF/FDM
            \item Tiskové materiály FFF
        \end{itemize}
    \end{frame}

    \begin{frame}
    \frametitle{O mně}

    \begin{columns}[T, onlytextwidth]
        \begin{column}[T]{0.6\textwidth}
            \begin{itemize}
                \item Absolvent GyBotu
                \item Bc. Kybernetika a robotika (FEL ČVUT)
                \item Ing. Umělá inteligence (FEL ČVUT)
                \item PhD. Počátačové vědy (FEL) - aktuálně
                \item 5. rok 3D tisk
                \begin{itemize}
                    \item Fandom předměty
                    \item Cosplay
                    \item Náhradní díly do domácnosti
                    \item CoVid pipetovací hlava
                    \item Pomůcky pro tělesně postižené
                    \item Díly pro roboty
                \end{itemize}
            \end{itemize}
            %\vspace{6em}
        \end{column}

        \begin{column}[T]{0.4\textwidth}
            \todotext add my photo
        \end{column}
        
    \end{columns}

\end{frame}

    \begin{frame}
    \frametitle{Motivace pro 3D tisk}
\end{frame}
    \begin{frame}
    \frametitle{Klasifikace}
\end{frame}
    
\begin{frame}
    \frametitle{Harmonogram seminářů}
    
\end{frame}
    \begin{frame}
    \frametitle{Technologie 3D tisku}
\end{frame}
    \begin{frame}
    \frametitle{Technogie FFF/FDM}
\end{frame}
    \begin{frame}
    \frametitle{Tiskové materiály}
\end{frame}

\end{document}
